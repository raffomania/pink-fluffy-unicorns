\documentclass{article}
\usepackage{amsfonts}
\usepackage[a4paper]{geometry}
\usepackage{alltt}
\usepackage{lmodern}
\usepackage{amssymb}
\usepackage{mathtools}
\usepackage{amsmath}
\usepackage{enumerate}
\usepackage{array}
\usepackage{listings}
\usepackage{fullpage}
\usepackage[parfill]{parskip}
\usepackage[utf8]{inputenc}
\usepackage[ngerman]{babel}

\title{Space Rave\\
    Documentation}
\author{Arne Beer, MN 6489196\\
Rafael Epplee, MN 6269560\\
Sven-Hendrik Haase, MN 6341873}

\begin{document}
\maketitle
    \section{Introduction}

    \section{Extracting rythm information from the sound file}

    \section{Modeling the Spaceship}
    We didn't use any modeling tool for making the spaceship. Instead, we opted for a blocky, ``minecrafty'' style with almost no details. We used unions to combine mostly boxes and prisms into quite a basic shape. Even the most complex part, the claws, are built with boxes and prisms.

    The size of the ship itself and various parts is easily customizable through local values declared at the top of \texttt{ship.inc}.

    Most parts of the ship are actually mirror-symmetric, like the base body, the claws and the window in the front. To minimize effort, we just modeled those parts in half and then mirrored them using duplication and scaling.

    The only non-symmetric parts of the ship are the engines at the rear end. They are each modeled on their own, each having a distinctive color.

    \section{Timing animations according to the song}

    \section{Summary}

\end{document}
